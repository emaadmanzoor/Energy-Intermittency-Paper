\documentclass[12pt,a4paper]{extarticle}
\usepackage[margin=1in]{geometry}
\usepackage[utf8]{inputenc}
\usepackage{booktabs} % for toprule, midrule and bottomrule
\usepackage{adjustbox}
\usepackage{amsmath}
\usepackage{bbold}
\usepackage{etoolbox}
\usepackage{setspace} % for \onehalfspacing and \singlespacing macros
\usepackage[hidelinks]{hyperref}
\usepackage{array}
\usepackage{graphicx}
\usepackage{setspace}
\usepackage{caption}
\usepackage{pdflscape}
\usepackage{caption}
\usepackage{tabularx}
\usepackage{authblk}
\usepackage{float}
\usepackage{siunitx}
\usepackage{titlesec}
\usepackage{pgfplots}
\usepackage[authoryear]{natbib}
\usepackage{scrextend}
\usepackage{nicefrac}
\usepackage{enumitem}
%\usepackage{showframe}
%\usepackage{lipsum}

% set space
%\doublespacing

%% section headings
%\renewcommand{\thesection}{\Roman{section}.\hspace{-0.5em}}
%\renewcommand\thesubsection{\Alph{subsection}.\hspace{-0.5em}}
%\renewcommand\thesubsubsection{\hspace{-1em}}
%\newcommand{\subsubsubsection}[1]{\begin{center}{\textit{#1}}\end{center}}
%
%\titleformat{\section}
%{\bf\centering\large}{\thesection}{1em}{}
%
%\titleformat{\subsection}
%{\itshape\centering}{\thesubsection}{1em}{}
%
%\titleformat{\subsubsection}
%{\bf}{\thesubsubsection}{1em}{}

% unicode chars for plots
\DeclareUnicodeCharacter{2212}{$-$}

% booktabs
\setlength\heavyrulewidth{0.06em} % 0.01em> midrule

% images
\graphicspath{ {D:/Users/saketh/Documents/GitHub/BECCS-Case-Study/documents/exhibits/} }

% array
\newcolumntype{L}[1]{>{\raggedright\let\newline\\\arraybackslash\hspace{0pt}}m{#1}}
\newcolumntype{C}[1]{>{\centering\let\newline\\\arraybackslash\hspace{0pt}}m{#1}}
\newcolumntype{R}[1]{>{\raggedleft\let\newline\\\arraybackslash\hspace{0pt}}m{#1}}

% caption set up
\captionsetup[table]{
	font = {sc},
	labelfont = {bf}
}

% sig stars
\def\sym#1{\ifmmode^{#1}\else\(^{#1}\)\fi}

% hyperlinks
\hypersetup{
	colorlinks=true,
	linkcolor = blue,
	urlcolor  = blue,
	citecolor = blue,
	anchorcolor = blue
}

% bibliography
\makeatletter
\renewenvironment{thebibliography}[1]
{\section*{References}%
	\@mkboth{\MakeUppercase\refname}{\MakeUppercase\refname}%
	\list{}%
	{\setlength{\labelwidth}{0pt}%
		\setlength{\labelsep}{0pt}%
		\setlength{\leftmargin}{\parindent}%
		\setlength{\itemindent}{-\parindent}%
		\@openbib@code
		\usecounter{enumiv}}%
	\sloppy
	\clubpenalty4000
	\@clubpenalty \clubpenalty
	\widowpenalty4000%
	\sfcode`\.\@m}
{\def\@noitemerr
	{\@latex@warning{Empty `thebibliography' environment}}%
	\endlist}
\makeatother

% etoolbox
\AtBeginEnvironment{quote}{\singlespacing}

\begin{document}
	
%	\title{\singlespacing{\textbf{%%%}}}
%	
%   \author[]{Saketh Aleti}
%	
%	\affil[]{\small{%%%}}
%	
%	\date{\vspace{-1em}\small{%%%}}
%	
%	\maketitle
	
	


%%%%%%%%%%%%%%%%%%%
%%%% OLD
%%%%%%%%%%%%%%%
%\section{Estimating the Elasticity of Substitution }
%\subsection{Methodology}
%
%Our CGE model uses CES production functions which must be supplied with estimates of the elasticity of substitution between commodities. We provide estimates of this parameter for our electricity nest which consists of electricity generated from biomass, geothermal, hydropower, solar, wind, nuclear, and fossil fuels. 
%
%Specifically, we construct two bundles from these sectors, renewable and non-renewable energy, and then estimate the elasticity of substitution within and between these bundles.\footnote{It should be noted that the IMPLAN data does not separate fossil fuel electricity generation by commodity; thus, due to a lack of data, we do not estimate a parameter for the fossil fuel bundle.} We first estimate the elasticity of substitution between renewable energy sources. To do so, we split the technologies into pairs: solar/wind, nuclear/biomass, and geothermal/hydropower. These pairs are chosen, because the technologies in each pair are similar in their intermittency and reliability. That is, solar and wind are both intermittent energy sources that cannot produce on demand. On the other hand, nuclear and biomass can generate output a consistent load but take a significant amount of time to start and stop. Thirdly, geothermal and hydropower both have consistent output and are relatively inexpensive to turn on and off. We estimate the elasticity of substitution for each pair and across all three pairs. 
%
%When estimating these parameters, we take a methodological approach similar to that of \citet{PSS}. That is, we estimate the CES function using NLS and then provide estimates of the \citet{Kmenta} approximation as a robustness check. Moreover, we assume neutral technological change to limit the number of parameters requiring identification. To start, we log the usual CES production function to obtain the following regression equation:
%\begin{equation}
%\ln Y_{it} = \alpha_i + \frac{1}{\phi} \ln \left( \sum_j \beta_{j} X^\phi_{ijt} \right) + \varepsilon_{it} 
%\end{equation}
%where $Y_{it}$ is the output in state $i$ at time $t$, $X_{ijt}$ is the capital input in state $i$ of commodity $j$ at time $t$, and $1/(1-\phi)$ is the elasticity of substitution. Taking a first-order Taylor series of this equation around $\phi = 0$, we find the Kmenta approximation:
%\begin{align}
%\ln Y_{it} &= \alpha_i + \sum_j \beta_{j} \ln ( X_{ijt}) + \sum_j \sum_{k } \gamma_{jk} \, \ln(X_{ijt}) \ln( X_{ikt}) \\
% &\text{where \quad} \phi_j =  \frac{2 \gamma_{jj}}{ \left( \beta_j^2 / \sum_k \beta_k \right) - \beta_j} 
%\end{align}
%The solution for $\phi$ is given by \citet{Anon}; this particular case corresponds to a CES with unitary returns to scale. This value of this approach lies in the fact that its parameters can be estimated directly with OLS. The primary drawback is that the estimates are less accurate when the true elasticity of substitution is far from $1$. Additionally, when estimating specification (2), we must constrain the coefficients such that (3) holds. Constrained OLS is necessary because, even if the underlying data conforms to a CES function, the Kmenta approximation may not exhibit CES properties when the underlying elasticity of substitution is not close to $1$; consequently, estimating without constraints may produce estimates of $\phi$ that vary with $i$ and $j$, which is not possible under a CES structure. 
%
%\subsection{Data}
%
%In order to estimate either function, we collect data on total electricity capacity and net generation for each source. We collect data on both using data from the US Energy Information Administration (EIA). Capacity data  is aggregated using data from Table 6.2.A and 6.2.B of the EIA's Electricity Power Monthly dataset, while net generation data is obtained from Table 1.3.A. Both sets of data are aggregated on a monthly basis for each state; we collect data for all states and every month in 2016. 


% Regress output in (MW/hr) on input (MW) capacity
% Capacity factors: https://www.eia.gov/electricity/monthly/epm_table_grapher.php?t=epmt_6_07_b
% Total capacity: https://www.eia.gov/electricity/monthly/epm_table_grapher.php?t=epmt_6_01
% Electricity net generation Mwh: https://www.eia.gov/electricity/state/









\pagebreak



	
\begin{thebibliography}{9}
	
	\bibitem[Anon(2004)]{Anon} 
	Anon (2004). The Linear Approximation of the CES Function with n Input Variables. (2004). Marine Resource Economics., 19(3), 295–306.
	
	\bibitem[Kmenta(1967)]{Kmenta} 
	Kmenta, J. (1967). On Estimation of the CES Production Function. International Economic Review, 8. Retrieved from http://search.proquest.com/docview/1299995933/
	
	\bibitem[Papageorgiou, Saarn, and Schulte(2017)]{PSS} 
	Papageorgiou, C., Saarn, M., \& Schulte, P. (2017). Substitution between clean and dirty energy inputs: a macroeconomic perspective. Review of Economics and Statistics, 99(2), 201–212. \href{https://doi.org/10.1162/REST_a_00592}{https://doi.org/10.1162/REST\_a\_00592}
		
%		\bibitem[Angel, Harris, and Spatt(2011)]{Angel} 
%		Angel, J.J., Harris, L.E., \& Spatt, C. S. (2011). Equity Trading in the 21st Century. Quarterly Journal Of Finance, 1(1), 1-53.
		
%		\bibitem[Stern(2012)]{Stern} 
%		Stern, D. (2012). “Interfuel Substitution: A Meta-Analysis.” Journal of Economic Surveys 26 (2): 307–331. \href{https://doi.org/10.1111/j.1467-6419.2010.00646.x}{https://doi.org/10.1111/j.1467-6419.2010.00646.x}.


		
\end{thebibliography}	
	
\end{document}





